%!TEX TS-program = xelatex
%!TEX encoding = UTF-8 Unicode

\documentclass[14pt, a4paper]{article}

% LAYOUT
%--------------------------------
\usepackage{geometry} 
\geometry{a4paper, left=35mm, right=60mm, top=30mm, bottom=30mm}

% No page numbers
\pagenumbering{gobble}

% Left align
\usepackage[document]{ragged2e}


  % To include the letterhead
  \usepackage{wallpaper}
  \ULCornerWallPaper{1}{letterhead.pdf}


% TYPOGRAPHY
%--------------------------------
\usepackage{fontspec} 
\usepackage{xunicode}
\usepackage{xltxtra}

% converts LaTeX specials (quotes, dashes etc.) to Unicode
\defaultfontfeatures{Mapping=tex-text}
\setromanfont [Ligatures={Common}, Numbers={OldStyle}]{Adobe Garamond Pro}
\setsansfont[Scale=0.9]{Adobe Garamond Pro}

% Set paragraph break
\setlength{\parskip}{1em}

% Custom ampersand
\newcommand{\amper}{{\fontspec[Scale=.95]{$seriffont$}\selectfont\itshape\&}}

$if(seriffont)$
\setmainfont[SmallCapsFeatures={LetterSpace=5,Letters=SmallCaps}]{$seriffont$}
$endif$
$if(sansfont)$
    \setsansfont{$sansfont$}
$endif$

% Command required by how Pandoc handles the list conversion
\providecommand{\tightlist}{%
  \setlength{\itemsep}{0pt}\setlength{\parskip}{0pt}}


% Rundung
%--------------------
 \usepackage{numprint}
 
% TABLE CUSTOMIZATION
%--------------------------------
\usepackage{spreadtab}
\usepackage[compact]{titlesec} % For customizing title sections
\titlespacing*{\section}{0pt}{3pt}{-7pt} % Remove margin bottom from the title
\usepackage{arydshln} % For the dotted line on the table
\renewcommand{\arraystretch}{1.5} % Apply vertical padding to table cells
\usepackage{hhline} % For single-cell borders
\usepackage{enumitem} % For customizing lists
\setlist{nolistsep} % No whitespace around list items
\setlist[itemize]{leftmargin=0.5cm} % Reduce list left indent
\setlength{\tabcolsep}{9pt} % Larger gutter between columns


% LANGUAGE
%--------------------------------
$if(lang)$
\usepackage{polyglossia}
\setmainlanguage{$lang$}
$endif$

% PDF SETUP
%--------------------------------
\usepackage[xetex, bookmarks, colorlinks, breaklinks]{hyperref}
\hypersetup
{
  pdfauthor=$author$,
  pdfsubject=Rechnung Nr. $invoice-nr$,
  pdftitle=Rechnung Nr. $invoice-nr$,
  linkcolor=blue,
  citecolor=blue,
  filecolor=black,
  urlcolor=blue
}

% To display custom date
% \usepackage[nodayofweek]{datetime}
% \newdate{date}{01}{12}{1867}
% \date{\displaydate{date}}
% Use this instead of \today: % \displaydate{date}

% DOCUMENT
%--------------------------------
\begin{document}
\today
\vspace{2em}

\normalsize \sffamily
$for(to)$
$to$\\
$endfor$

\vspace{3em}



\vspace{1em}


\section*{Rechnung \#$invoice-nr$}

\vspace{10mm}

\sffamily\small $ansprache$

\medskip

\sffamily
\small
$startnote$
\medskip

\footnotesize
%\STautoround*{2} % Get spreadtab to always display the decimal part
\nprounddigits{2}
$if(commasep)$\STsetdecimalsep{,}$endif$ % Use comma as decimal separator

\begin{spreadtab}{{tabular}{cp{4cm}N{8}{2}N{8}{2}}}
  \hdashline[1pt/1pt]
  @ \noalign{\vskip 2mm} \textbf{Menge} & @ \textbf{Beschreibung} & @ \textbf{Einzelpreis in $currency$} & @ \textbf{Gesamtpreis in $currency$} \\ \hline
      $for(service)$  $service.count$ 
        & @ $service.description$ 
        $if(service.details)$\newline \begin{itemize} 
          $for(service.details)$\scriptsize \item $service.details$ 
          $endfor$ \end{itemize}
          $endif$ &  $service.price$ & :={[-3,0]*[-1,0]} \\$endfor$ \noalign{\vskip 2mm} \hline
          $if(VAT)$
    @ & @ & @ \multicolumn{1}{r}{Nettobetrag:}                & :={sum(d1:[0,-1])} \\ \hhline{~~~-}
    @ & @ & @ \multicolumn{1}{r}{USt $VAT$\%:}               & $VAT$/100*[0,-1] \\ \hhline{~~~-}
 $endif$
    @ & @ & @ \multicolumn{1}{r}{Gesamtbetrag:}                & :={sum(d1:[0,-1])} \\ \hhline{~~~-}
\end{spreadtab}


\vspace{7mm}

\sffamily
\small
$middelnote$

\medskip

$firm$\\
IBAN: $IBAN$\\
BIC: $BIC$

\medskip

$closingnote$

\medskip

$author$\\
$firm$

\end{document}
